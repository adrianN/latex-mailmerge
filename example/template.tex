\documentclass[a4paper,10pt]{article}
\usepackage[landscape,left=3cm,right=3cm]{geometry}
\usepackage[utf8]{inputenc}
\usepackage[T1]{fontenc}
\usepackage{graphicx}
\usepackage{palatino}
\usepackage{scalefnt}

\setlength{\parindent}{0pt}

\newcommand{\LectureTitle}{Example Lecture}
\newcommand{\Lecturers}{Smart Professor \and Clever Assistant}
\newcommand{\LectureType}{Lecture}
\newcommand{\CreditPoints}{9}

\begin{document}
  \thispagestyle{empty}
  \begin{samepage}
	%\includegraphics[height=2cm]{mpii-presentation-logo}%
	%\hfill
	%\includegraphics[height=2cm]{uds-cs-logo}%
	\vspace*{2cm}

    \begin{center}
      \scalefont{3} \textbf{\LectureTitle}
    \end{center}

    \begin{center}
      \scalefont{2}\bfseries
      \lineskip .75em%
      \begin{tabular}[t]{c}%
        \Lecturers
      \end{tabular}%
    \end{center}

    \vspace{1cm}
    \begin{samepage}
	\scalefont{1.5}
	%Simple Python expression can be used
	\begin{python}Name\end{python}, \begin{python}Firstname\end{python} (Matriculation number \begin{python}Matriculation\end{python}) attended the \LectureType\bigskip
	\begin{center}
	 \textbf{\LectureTitle}
	\end{center}\bigskip
	with \CreditPoints{} CP \begin{python}'successfully' if float(Grade)<4.5 else ''\end{python}.\\
	\begin{python}'She' if Gender=='female' else 'He'\end{python} got the grade
	\begin{center}
	 \begin{python}"%(x).1f" % {"x":float(Grade)}\end{python} (%
	 %if the Python code gets too complicated for simple statements
	 %you can also assign things to the magic variable "text"
	 \begin{python}
gr = float(Grade)
if gr<1.5:
  text = "Great"
else:
  text = "okay"
	 \end{python}%
	 )
	\end{center}
	\vspace{3cm plus 0.7cm minus 0.7cm}

    \newcommand{\layoutedLecturer}{\begingroup\renewcommand{\and}{\& }\footnotesize{\Lecturers}\endgroup}%
    \hspace*{0pt plus 4fill}%
    \begin{minipage}[b][1cm]{10cm}\normalsize
      \today\\[-1ex]
      \rule{\linewidth}{.5pt}\\
      \footnotesize\strut\hfill\layoutedLecturer%
    \end{minipage}%
    \end{samepage}
  \end{samepage}
\end{document}